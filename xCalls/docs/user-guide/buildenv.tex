\section{Build Process}

\subsection{Building Library}
TxLib is built by invoking the SCons software construction tool in   
the \verb!$XCALLS_HOME! directory.    

\begin{verbatim}
% scons
\end{verbatim}

\noindent The built process can be parameterized using the following set of arguments:

\begin{itemize}
\item \verb!mode=[debug, release]! Builds either a library with debugging
support or an optimized version for use in productivity environments.  
\item \verb!linkage=[dynamic, static]! Builds either a dynamic shared library (.so)
 or a static archive (.a).  
\item \verb!stats=[enable,disable]! Builds the library with profiling support.   
\item \verb!test=[disable,enable]! Builds the library and then tests the library 
using the unit tests found under \verb!$XCALLS_HOME!/tests   
\end{itemize}   

The first option is the default one. 

\subsection{Building Applications}
Makefiles used to build applications using the xCalls library should be 
modified in accordance to the guidelines presented here. 

\subsubsection{Symbol definitions}
The following symbols must be defined when invoking the compiler:
\begin{itemize}
\item \verb!_GNU_SOURCE!
\item \verb!TXC_XCALLS_ENABLE!
\item \verb!TM_CALLABLE="__attribute__ ((tm_callable))"!
\item \verb!TM_WAIVER="__attribute__ ((tm_pure))"!
\end{itemize}

\paragraph{Invocation example}
\begin{verbatim}
icc  -Qtm_enabled -D_GNU_SOURCE -DTXC_XCALLS_ENABLE 
-DTM_CALLABLE="__attribute__ ((tm_callable))" 
-DTM_WAIVER="__attribute__ ((tm_pure))" 
\end{verbatim}

\subsubsection{Header files}
The xCall interface is available through the following header file:
\begin{itemize}
\item \verb!txc.h!
\end{itemize}

\noindent These files are available in 
\verb!$XCALLS_HOME/include!

\paragraph{Invocation example}
\begin{verbatim}
icc -I txc/include
\end{verbatim}

\paragraph{Important Note} The icc compiler by default uses built-in versions
of common function such as {\tt printf, memcpy}, etc called {\em intrinsic} functions. 
This also includes the transactional version of memory allocation routines 
({\tt malloc, realloc, calloc, free}).
However, we have noticed that the compiler might generate incorrect code when 
using the built-in versions inside transactions. For example, we came across
a case where the compiler would silently without any warning message inline 
{\tt memset} inside a transaction but incorrectly
not instrument the memory references produced by the inlining. 
To avoid such unexpected problems we recommend passing the {\tt -fno-builtin} option
to the compiler. However this has the side effect that the compiler does not use
the transactional version of memory allocators. TxLib provides wrappers that use 
the internal transactional memory allocators. See section \ref{section:xcalls:memalloc} 
for more information.      
