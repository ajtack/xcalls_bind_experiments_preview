\section{Programming Interface}


\subsection{TM Interface}

The Transactional Memory (TM) interface provides wrapper macros and functions for 
defining and manipulating transactions which allow the safe invocation of \xcalls. 
This interface is intended to be used both by the application and \xcalls
library programmer.

\paragraph {Wrapper Macros}:\newline
\begin{tabular*}{12cm}{p{4cm}p{8cm}}
\verb!XACT_BEGIN(tag)! 	& Begins a transaction block. The \texttt{tag} field is 
mandatory and it is used to name the transaction block. \\
\verb!XACT_END(tag)! 		& Ends a transaction block \\
\verb!XACT_RETRY! 			&	Forces abort and restart of the current transaction.  \\
\verb!GET_TRANSACTION_MODE! \verb!(xact_mode)! &  Returns the transactional mode 
of the active thread	(non-transactional, optimistic/pessimistic transaction, 
irrevocable transaction)	\\
\verb!XACT_SWITCH_TO_! \verb!IRREVOCABLE!  & Forces the current transaction to execute 
in \xcalls~irrevocable mode.	\\
\end{tabular*}
\newline
\newline
Since \verb!XACT_BEGIN! 
and \verb!XACT_END! wrap the \tmatomic~keyword, they must be lexically scoped.
For example the following is illegal:

\begin{verbatim}

__attribute__ ((tm_callable)) 
void goo() {
  ...
  XACT_END(xact_foo)
}

void foo() {
  XACT_BEGIN(xact_foo)
  goo();
}

\end{verbatim}


\paragraph{Functions}:\newline
\verb!void txc_global_init()! Initializes the TM and xCalls system. Must be 
called before using any of the xCalls in an application.
\newline
\noindent\verb!void txc_thread_init()! Initializes the xCalls transaction 
descriptor associated with each thread in an application. A thread must call it
before using any of the xCalls.

\begin{verbatim}
void txc_register_commit_action(
         txc_result_t (*action)(unsigned int, unsigned int *), 
         unsigned int num_args, 
         unsigned int *arg_list)
\end{verbatim}
Registers function \texttt{action} to be called when the transaction commits.
The function accepts a list of parameters in array form which is passed to 
function \texttt{action} when invoking it on commit.
\texttt{action} accepts two arguments; the first is the length of the parameter 
list and the second the list in array form. \textit{Note}: Currently the parameters 
can be only of integral type. 


\begin{verbatim}
void txc_register_compensating_action(
         txc_result_t (*action)(unsigned int, unsigned int *), 
         unsigned int num_args, 
         unsigned int *arg_list)
\end{verbatim}
Registers function \texttt{action} to be called when the transaction aborts.
The function accepts a list of parameters in array form which is passed to 
function \texttt{action} when invoking it on abort.

\paragraph{Usage Example}
\begin{verbatim}
#include <txc/transaction.h>
#include <txc/wrappers.h>

void child_main(void *arg)
{
  txc_global_init();

  XACT_BEGIN(xact_child)

  XACT_END(xact_child)
}

void main()
{
  txc_thread_init();

  pthread_create(&child, NULL, child_main, NULL);
}
\end{verbatim}

\subsection{Sentinels Interface}

The sentinel interface provides macros for manipulating the sentinel
abstraction.
This interface is intended to be used by the \xcalls~library programmer.

\begin{tabular*}{12cm}{ll}
\verb!SENTINEL_ACQUIRE_EXCLUSIVE(sentinel)! & Acquires sentinel in exclusive mode. \\
\verb!SENTINEL_ACQUIRE_SHARED(sentinel)! & Acquires sentinel in shared mode.			\\
\verb!SENTINEL_RELEASE(sentinel)! & Releases sentinel.
\end{tabular*}


\subsection{xCalls Interface}

This interface is intended to be used by the application programmer.
Its description is given in the document "xCalls: Safe I/O in Memory Transactions"

\subsubsection{Memory Allocation}
When passing the {\tt -fno-builtin} option to the compiler, the compiler does not use
the transactional version of memory allocators. To resolve this issue, TxLib provides 
the wrappers {\tt txc\_malloc, txc\_realloc, txc\_calloc, txc\_free} which use 
the internal transactional memory allocators provided by the STM. 
