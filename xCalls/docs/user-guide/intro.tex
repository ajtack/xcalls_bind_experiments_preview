\section{Introduction}

\subsection{xCalls}
The xCall interface enables  executing I/O and system calls
within memory transactions. Similar to database transactions, memory
transactions allow a programmer focus on the logic of their program
and let the system ensure that transactions are atomic and
isolated. However, most transactional memory systems only support
transactions within a user-mode process. When a transaction performs
I/O or accesses kernel resources, the atomicity and isolation guarantees
from the TM system do not apply to the kernel.

The xCall interface provides isolation and atomicity for kernel data
objects with a new programming API for making system calls within
transactions. Rather than making every system call transactional, the xCall API
handles the common cases of file access, communication, and
synchronization. xCalls provide isolation for kernel resources with
sentinels, which are revocable lightweight user-level locks. A
transaction acquires a sentinel exclusively when it accesses kernel
resource, such as a file, through a system call. Competing threads
must block until the transaction completes and release the
sentinel. xCalls provide atomicity for system calls through a
combination of {\em deferral}, delaying execution until the
transaction commits, and {\em compensation}, calling back into the
kernel to undo the side effects of a previous call. However, the xCall
interface specifies {\em when} every call executes, so programmers are
aware when the side effects of an xCall become visible. Finally,
xCalls return errors asynchronously, after the transaction completes,
when a deferred system call or compensation fails.

\subsection{Availability}
Currently the xCall interface is compatible with the Intel C/C++ software
transactional memory (STM) compiler prototype edition 2.0. The source 
code of the library is available through the SVN repository:
\newline
\begin{tabular}{ll}
Repository Root: & \verb!/p/multifacet/projects/naxos/NAXOS_SVN_ROOT! \\
Project: & \verb!xCalls/txc/v2/!
\end{tabular}
